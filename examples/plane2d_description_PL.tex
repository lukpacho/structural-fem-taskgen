% plane2d_template.tex
\documentclass[a4paper,10pt]{article}
\usepackage[left=20mm,right=20mm,top=25mm,bottom=25mm,bindingoffset=0mm]{geometry}
\usepackage{fontspec}
\usepackage{fancyhdr}
\usepackage{graphicx}
\usepackage[detect-all=true]{siunitx}
\usepackage{stanli}
\usepackage{amsmath}
\usepackage{booktabs}
\usepackage{tikz}
\usepackage{calc}
\usepackage{xcolor}
\usepackage{adjustbox}
\usepackage{enumitem} 


\pagestyle{fancy}
\fancyhead[L]{\small Katedra Wytrzymałości Materiałów, WILiŚ, PG}
\fancyhead[R]{\small rok. akad. 2025/2026}
\fancyfoot[L]{\textcolor{gray}{\textit{Numer zadania (do informacji sprawdzającego): 482000}}}
\fancyfoot[C]{}


% Document
\begin{document}

\begin{center}
    {\Large \textbf{Podstawy Mechaniki Komputerowej - Projekt}}
    \end{center}

    \begin{table}[ht]
        \centering
        \begin{tabular}{
            m{.175\textwidth}m{.215\textwidth}m{.175\textwidth}m{.175\textwidth}m{.1\textwidth}}
        \toprule
        Imię & Nazwisko & Numer albumu & Numer grupy  & Ocena \\ \midrule
             &          &              &              &       \\ \bottomrule
        \end{tabular}
        \label{tab:dane_studenta}
    \end{table}

    \noindent\textbf{Treść zadania:} \vspace{1mm}

    \noindent Dla przedstawionego układu w \textbf{Płaskin Stanie Odkształcenia (PSO)} wyznaczyć:
    \begin{itemize}
        \item wartości \textbf{przemieszczeń poziomych} $(u_x)$ w węźle 7 oraz \textbf{pionowych} $(v_y)$ w węźle 8,
        \item \textbf{maksymalną} \textbf{bezwzględną} wartość naprężeń normalnych $\mathrm{max}(|\sigma_x|)$ oraz ich \textbf{lokalizację}.
    \end{itemize}

    \noindent Zadanie należy rozwiązać stosując trójkątny element skończony (CST) dla: \textbf{A)} siatki przedstawionej na rysunku, \textbf{B)} siatki wygenerowanej automatycznie przy parametrze $H_{\mathrm{max}} = 0.36$\,$\si{m}$. \\
    Konstrukcja ma grubość $t = 80$\,$\si{m}$ i wykonano ją z materiału o parametrach $E = 12$\,$\si{GPa}$ oraz $\nu = 0.25$. \\
    Zadany układ podparty jest na krawędzi pomiędzy węzłami:
        5 - 6 w obu kierunkach%
,         10 - 1 w obu kierunkach%
. 
    Na układ działają obciążenia w postaci 
 sił skupionych: 
        $P_1 = 14400\,\si{kN}$ w węźle 3 na kierunku pionowym%
,         $P_2 = 6400\,\si{kN}$ w węźle 9 na kierunku pionowym%
. \newline

    \noindent\textbf{Schemat układu} (wymiary w metrach):
    
    \includegraphics[width=\textwidth]{/media/lukpacho/Disk-1/OneDrivePG/PycharmProjects/structural-fem-taskgen/out/temp/plane2_0_mesh.pdf}\par

    \noindent\textbf{Wymagane elementy opracowania:}

    \begin{enumerate}
    \item Skrypt do rozwiązania części \textbf{A)} \dotfill 10 pkt
    \item Skrypt do rozwiązania części \textbf{B)} \dotfill 10 pkt
    \item Wyniki przemieszczeń i naprężeń (w wybranych jednostkach) w tabeli poniżej oraz lokalizacje naprężeń oznaczone na rysunku (dla obu wariantów) \dotfill 5 pkt
    \begin{table}[ht]
        \centering
        \begin{tabular}{@{}m{.02\textwidth} m{.15\textwidth} m{.15\textwidth} m{.2\textwidth} @{}}
        \toprule
         & $u_x$ [\dotfill] & $v_y$ [\dotfill] & $\mathrm{max}(|\sigma_x|)$ [\dotfill] \\ \midrule
         \textbf{A)} & & & \\ \midrule
         \textbf{B)} & & & \\ \bottomrule
        \end{tabular}
        \label{tab:rezultaty}
    \end{table}
    \end{enumerate}

    \noindent\textbf{Uwagi:}
    \begin{itemize}
    \item W analizie numerycznej należy wykorzystać bibliotekę \textbf{CalFEM} w środowisku \textbf{MATLAB}.
    \item Skrypty rozwiązujące zadanie należy przesłać do odpowiedniego modułu na platformie eNauczanie w ramach kursu \textit{Podstawy Mechaniki Komputerowej}.
    \item Nieprzesłanie skryptów skutkuje przyznaniem 0 punktów za całe zadanie.
    \end{itemize}

\end{document}