\documentclass[a4paper,10pt]{article}
\usepackage[left=20mm,right=20mm,top=25mm,bottom=25mm,bindingoffset=0mm]{geometry}
\usepackage{fontspec}
\usepackage{fancyhdr}
\usepackage{graphicx}
\usepackage[detect-all=true]{siunitx}
\usepackage{stanli}
\usepackage{amsmath}
\usepackage{booktabs}
\usepackage{tikz}
\usepackage{calc}
\usepackage{xcolor}
\usepackage{adjustbox}


\pagestyle{fancy}
\fancyhead[L]{\small Katedra Wytrzymałości Materiałów, WILiŚ, PG}
\fancyhead[R]{\small rok. akad. \VAR{academic_year}}
%\fancyfoot[L]{\textcolor{gray}{\textit{Numer zadania (do informacji sprawdzającego): \VAR{beam_version_num_hidden}\VAR{simulation_index}}}}
\fancyfoot[L]{\textcolor{gray}{\textit{Numer zadania (do informacji sprawdzającego): plane\_version\_num\_hidden simulation\_index}}
\fancyfoot[C]{}


% Document
\begin{document}

\begin{center}
    {\Large \textbf{Podstawy Mechaniki Komputerowej - Projekt}}
    \end{center}

    \begin{table}[ht]
        \centering
        \begin{tabular}{
            m{.175\textwidth}m{.215\textwidth}m{.175\textwidth}m{.175\textwidth}m{.1\textwidth}}
        \toprule
        Imię & Nazwisko & Numer albumu & Numer grupy  & Ocena \\ \midrule
             &          &              &              &       \\ \bottomrule
        \end{tabular}
        \label{tab:dane_studenta}
    \end{table}

    \noindent\textbf{Treść zadania:} \vspace{1mm}

    Dla przedstawionego układu prętowego należy:

    \begin{itemize}
        \item Wyznaczyć wykresy sił wewnętrznych: sił tnących \(T\) oraz momentów zginających \(M\), \textbf{przedsta\-wione w odpowiednich jednostkach}.
        \item Sporządzić \textbf{rysunek} deformacji układu, oznaczając \textbf{wartości charakterystyczne ugięć} oraz uwzględniając \textbf{obliczone kąty obrotu}.
    \end{itemize}

    Zadanie należy rozwiązać, stosując trójkątny element skończony (CST) w \textbf{Płaskim Stanie Naprężenia (PSN)/Płaskim Stanie Odkształcenia (PSO)}. W analizie numerycznej proszę wykorzystać bibliotekę \textbf{CalFEM} w środowisku \textbf{MATLAB}. Poniższa tabela zawiera dane niezbędne do realizacji zadania:

    \begin{table}[ht]
    \centering
    \renewcommand{\arraystretch}{1.25}
    \begin{tabular}{lll}
    \toprule
    Długość [\si{m}] &
            \\
    Moduł Younga [\si{GPa}] &
            \\
    Moment bezwładności [\si{\centi\meter^4}] &
            \\
    \bottomrule
    \end{tabular}
    \label{tab:dane_zadania2}
    \end{table}

    \vspace{5mm}

    \noindent\textbf{Schemat układu:}

    \vspace{5mm}

    \noindent\textbf{Wymagane elementy opracowania:}

    \begin{enumerate}
    \item Rysunek dyskretyzacji układu \dotfill 5 pkt
    \item Skrypt w MATLAB-ie rozwiązujący zadanie (*.m) \dotfill 5 pkt
    \item Wykresy sił wewnętrznych: siły tnące \(T\) [\si{\kilo\newton}] i momenty zginające \(M\) [\si{\kilo\newton m}] \dotfill 10 pkt
    \item Rysunek deformacji układu [\si{\centi\meter}] \dotfill 5 pkt
    \end{enumerate}

    \vspace{5mm}

    \noindent\textbf{Uwagi:}
    \begin{itemize}
    \item Rysunek dyskretyzacji, wykresy sił tnących, momentów zginających oraz rysunek deformacji należy umieścić na odwrocie karty projektowej.
    \item Skrypt rozwiązujący zadanie należy przesłać do odpowiedniego modułu na platformie eNauczanie w ramach kursu \textit{Podstawy Mechaniki Komputerowej}.
    \item Nieprzesłanie skryptu skutkuje przyznaniem 0 punktów za elementy opracowania nr 3 i 4.
    \end{itemize}

\end{document}