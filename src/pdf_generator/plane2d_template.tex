\documentclass[a4paper,10pt]{article}
\usepackage[left=20mm,right=20mm,top=25mm,bottom=25mm,bindingoffset=0mm]{geometry}
\usepackage{fontspec}
\usepackage{fancyhdr}
\usepackage{graphicx}
\usepackage[detect-all=true]{siunitx}
\usepackage{stanli}
\usepackage{amsmath}
\usepackage{booktabs}
\usepackage{tikz}
\usepackage{calc}
\usepackage{xcolor}
\usepackage{adjustbox}


\pagestyle{fancy}
\fancyhead[L]{\small Katedra Wytrzymałości Materiałów, WILiŚ, PG}
\fancyhead[R]{\small rok. akad. \VAR{academic_year}}
\fancyfoot[L]{\textcolor{gray}{\textit{Numer zadania (do informacji sprawdzającego): \VAR{plane2d_version_num_hidden}\VAR{simulation_index}}}}
\fancyfoot[C]{}


% Document
\begin{document}

\begin{center}
    {\Large \textbf{Podstawy Mechaniki Komputerowej - Projekt}}
    \end{center}

    \begin{table}[ht]
        \centering
        \begin{tabular}{
            m{.175\textwidth}m{.215\textwidth}m{.175\textwidth}m{.175\textwidth}m{.1\textwidth}}
        \toprule
        Imię & Nazwisko & Numer albumu & Numer grupy  & Ocena \\ \midrule
             &          &              &              &       \\ \bottomrule
        \end{tabular}
        \label{tab:dane_studenta}
    \end{table}

    \noindent\textbf{Treść zadania:} \vspace{1mm}

    Dla przedstawionego układu w \textbf{\VAR{analysis_type}} należy:
    \begin{itemize}
        \item Wyznaczyć wartości przemieszczeń poziomych $u$ oraz pionowych $v$, w oznaczonych miejscach na rysunku.
        \item Wyznaczyć \textbf{maksymalną} \textbf{bezwzględną} wartość naprężeń normalnych $max(|\sigma|)$ ($\sigma_x$ lub $\sigma_y$).
    \end{itemize}

    Zadanie należy rozwiązać, stosując trójkątny element skończony (CST) zarówno dla \textbf{A)} siatki przedstawionej na rysunku, jak i dla \textbf{B)} siatki wygenerowanej automatycznie przy parametrze $H_{max} = \VAR{el_size_factor} \si{m}$.  Konstrukcja ma grubość $t = \VAR{t} \si{mm}$ i wykonano ją z materiału o parametrach $E = \VAR{E} \si{GPa}$ oraz $\nu = \VAR{nu}$. Zadany układ podparty jest na krawędzi pomiędzy węzłami
    \BLOCK{ for bc in bc_list}
        \VAR{bc.point1} oraz \VAR{bc.point2} na \VAR{bc.direction}%
    \BLOCK{ endfor }. 
    Działają na niego obciążenia w postaci 
    \BLOCK{ for force in forces}
        \BLOCK{if loop.first}
            \BLOCK{ if loop.length > 1} sił skupionych: 
            \BLOCK{ else } siły skupionej
            \BLOCK{ endif }
        \BLOCK{ endif }
        $P_\VAR{loop.index} = \VAR{force.value} \si{\kilo\newton}$ w węźle \VAR{force.point} na kierunku \VAR{force.direction}%
        \BLOCK{ if not loop.last }, \BLOCK{ endif }
    \BLOCK{ endfor }.

    \vspace{5mm}

    \noindent\textbf{Schemat układu} (wymiary w metrach):

    \vspace{5mm}

    \begin{figure}[ht]
        \centering
        \includegraphics[width=0.8\textwidth]{\VAR{plot_path}}
    \end{figure}

    \noindent\textbf{Wymagane elementy opracowania:}

    \begin{enumerate}
    \item Rysunek dyskretyzacji układu \dotfill 5 pkt
    \item Skrypt w MATLAB-ie rozwiązujący zadanie (*.m) \dotfill 5 pkt
    \item Wykresy sił wewnętrznych: siły tnące \(T\) [\si{\kilo\newton}] i momenty zginające \(M\) [\si{\kilo\newton m}] \dotfill 10 pkt
    \item Rysunek deformacji układu [\si{\centi\meter}] \dotfill 5 pkt
    \end{enumerate}

    \vspace{5mm}

    \noindent\textbf{Uwagi:}
    \begin{itemize}
    \item W analizie numerycznej należy wykorzystać bibliotekę \textbf{CalFEM} w środowisku \textbf{MATLAB}.
    \item Skrypt/skrypty rozwiązujące zadanie należy przesłać do odpowiedniego modułu na platformie eNauczanie w ramach kursu \textit{Podstawy Mechaniki Komputerowej}.
    \item Nieprzesłanie skryptu skutkuje przyznaniem 0 punktów za elementy opracowania nr 3 i 4.
    \end{itemize}

\end{document}