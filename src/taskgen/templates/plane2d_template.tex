% plane2d_template.tex
\documentclass[a4paper,10pt]{article}
\usepackage[left=20mm,right=20mm,top=25mm,bottom=25mm,bindingoffset=0mm]{geometry}
\usepackage{fontspec}
\usepackage{fancyhdr}
\usepackage{graphicx}
\usepackage[detect-all=true]{siunitx}
\usepackage{stanli}
\usepackage{amsmath}
\usepackage{booktabs}
\usepackage{tikz}
\usepackage{calc}
\usepackage{xcolor}
\usepackage{adjustbox}
\usepackage{enumitem} 


\pagestyle{fancy}
\fancyhead[L]{\small Katedra Wytrzymałości Materiałów, WILiŚ, PG}
\fancyhead[R]{\small rok. akad. \VAR{academic_year}}
\fancyfoot[L]{\textcolor{gray}{\textit{Numer zadania (do informacji sprawdzającego): \VAR{plane2d_version_num_hidden}\VAR{simulation_index}}}}
\fancyfoot[C]{}


% Document
\begin{document}

\begin{center}
    {\Large \textbf{Podstawy Mechaniki Komputerowej - Projekt}}
    \end{center}

    \begin{table}[ht]
        \centering
        \begin{tabular}{
            m{.175\textwidth}m{.215\textwidth}m{.175\textwidth}m{.175\textwidth}m{.1\textwidth}}
        \toprule
        Imię & Nazwisko & Numer albumu & Numer grupy  & Ocena \\ \midrule
             &          &              &              &       \\ \bottomrule
        \end{tabular}
        \label{tab:dane_studenta}
    \end{table}

    \noindent\textbf{Treść zadania:} \vspace{1mm}

    \noindent Dla przedstawionego układu w \textbf{\VAR{analysis_type}} wyznaczyć:
    \begin{itemize}
        \item wartości \textbf{przemieszczeń poziomych} $(u_x)$ w węźle \VAR{max_u_node} oraz \textbf{pionowych} $(v_y)$ w węźle \VAR{max_v_node},
        \item \textbf{maksymalną} \textbf{bezwzględną} wartość naprężeń normalnych $\mathrm{max}(|\VAR{max_abs_stress}|)$ oraz ich \textbf{lokalizację}.
    \end{itemize}

    \noindent Zadanie należy rozwiązać stosując trójkątny element skończony (CST) dla: \textbf{A)} siatki przedstawionej na rysunku, \textbf{B)} siatki wygenerowanej automatycznie przy parametrze $H_{\mathrm{max}} = \VAR{el_size_factor}$\,$\si{m}$. \\
    Konstrukcja ma grubość $t = \VAR{t}$\,$\si{\VAR{t_unit}}$ i wykonano ją z materiału o parametrach $E = \VAR{E}$\,$\si{GPa}$ oraz $\nu = \VAR{nu}$. \\
    Zadany układ podparty jest na krawędzi pomiędzy węzłami:
    \BLOCK{ for bc in bc_list}
        \VAR{bc.point1} - \VAR{bc.point2} \VAR{bc.direction}%
        \BLOCK{ if not loop.last }, \BLOCK{ endif }
    \BLOCK{ endfor }. 
    Na układ działają obciążenia w postaci 
    \BLOCK{ for force in forces}
        \BLOCK{if loop.first}
            \BLOCK{ if loop.length > 1} sił skupionych: 
            \BLOCK{ else } siły skupionej
            \BLOCK{ endif }
        \BLOCK{ endif }
        $P_\VAR{loop.index} = \VAR{force.value}\si{kN}$ w węźle \VAR{force.point} na kierunku \VAR{force.direction}%
        \BLOCK{ if not loop.last }, \BLOCK{ endif }
    \BLOCK{ endfor }. \newline

    \noindent\textbf{Schemat układu} (wymiary w metrach):
    
    \includegraphics[width=\textwidth]{\VAR{plot_path}}\par

    \noindent\textbf{Wymagane elementy opracowania:}

    \begin{enumerate}
    \item Skrypt do rozwiązania części \textbf{A)} \dotfill 10 pkt
    \item Skrypt do rozwiązania części \textbf{B)} \dotfill 10 pkt
    \item Wyniki przemieszczeń i naprężeń (w wybranych jednostkach) w tabeli poniżej oraz lokalizacje naprężeń oznaczone na rysunku (dla obu wariantów) \dotfill 5 pkt
    \begin{table}[ht]
        \centering
        \begin{tabular}{@{}m{.02\textwidth} m{.15\textwidth} m{.15\textwidth} m{.2\textwidth} @{}}
        \toprule
         & $u_x$ [\dotfill] & $v_y$ [\dotfill] & $\mathrm{max}(|\VAR{max_abs_stress}|)$ [\dotfill] \\ \midrule
         \textbf{A)} & & & \\ \midrule
         \textbf{B)} & & & \\ \bottomrule
        \end{tabular}
        \label{tab:rezultaty}
    \end{table}
    \end{enumerate}

    \noindent\textbf{Uwagi:}
    \begin{itemize}
    \item W analizie numerycznej należy wykorzystać bibliotekę \textbf{CalFEM} w środowisku \textbf{MATLAB}.
    \item Skrypty rozwiązujące zadanie należy przesłać do odpowiedniego modułu na platformie eNauczanie w ramach kursu \textit{Podstawy Mechaniki Komputerowej}.
    \item Nieprzesłanie skryptów skutkuje przyznaniem 0 punktów za całe zadanie.
    \end{itemize}

\end{document}