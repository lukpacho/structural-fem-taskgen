\documentclass[a4paper,10pt]{article}
\usepackage[left=25mm,right=25mm,top=25mm,bottom=25mm,bindingoffset=0mm]{geometry}
\usepackage{fontspec}
\usepackage{fancyhdr}
\usepackage{graphicx}
\usepackage[detect-all=true]{siunitx}
\usepackage{stanli}
\usepackage{amsmath}
\usepackage{booktabs}

\pagestyle{fancy}
\fancyhead[L]{\small Katedra Wytrzymałości Materiałów, WILiŚ, PG}
\fancyhead[R]{\small rok. akad. \VAR{academic_year}}
\fancyfoot{}

\begin{document}

\begin{center}
    {\Large \textbf{Metody Obliczeniowe - Projekt}}
\end{center}

\begin{table}[ht]
    \centering
    \begin{tabular}{
        m{.175\textwidth}m{.215\textwidth}m{.175\textwidth}m{.175\textwidth}m{.1\textwidth}}
    \toprule
    Imię & Nazwisko & Numer albumu & Numer grupy  & Ocena \\ \midrule
         &          &              &              &       \\ \bottomrule
    \end{tabular}
    \label{tab:dane_studenta}
\end{table}

Dla danego układu prętowego należy wyznaczyć wykresy sił wewnętrznych (siły tnące \(T\) i momenty zginające \(M\))
oraz przedstawić szkic deformacji układu. Zadanie należy rozwiązać, stosując Macierzową Metodę Przemieszczeń
dla elementów belkowych. W analizie numerycznej proszę wykorzystać bibliotekę CalFEM w środowisku MATLAB.
Poniższa tabela zawiera dane wymagane do realizacji zadania, przedstawione w jednostkach:
\(L \rightarrow \si{m}\), \(E \rightarrow \si{GPa}\), \(I \rightarrow \si{\centi\meter^4}\).

%\begin{minipage}[s]{\textwidth}
%    \vspace{3mm}
%    \renewcommand{\arraystretch}{1.25}
%    \centering
%    \begin{tabular}{llllll}
%        \multicolumn{6}{l}{Zestaw danych nr \VAR{beam_version}\VAR{simulation_index}} \\
%        $L_1$ = \VAR{lengths[0]} & $L_1$ = \VAR{lengths[1]} & $L_1$ = \VAR{lengths[2]} & $L_1$ = \VAR{lengths[3]} & $L_1$ = \VAR{lengths[4]} & $L_1$ = \VAR{lengths[5]} \\
%        E1 = \VAR{materials[0]} & E2 = \VAR{materials[1]} & E3 = \VAR{materials[2]} & E4 = \VAR{materials[3]} & E5 = \VAR{materials[4]} & E6 = \VAR{materials[5]} \\
%        I1 = \VAR{inertia[0]} & I2 = \VAR{inertia[1]} & I3 = \VAR{inertia[2]} & I4 = \VAR{inertia[3]} & I5 = \VAR{inertia[4]} & I6 = \VAR{inertia[5]}
%    \end{tabular}
%    \label{tab:dane_zadania}
%    \vspace{3mm}
%\end{minipage}

\noindent
\begin{minipage}[s]{\textwidth}
    \vspace{3mm}
    \renewcommand{\arraystretch}{1.25}
    \centering
    \begin{tabular}{\VAR{"l" * (nels)}}
        \multicolumn{\VAR{nels}}{l}{Zestaw danych nr \VAR{beam_version}\VAR{simulation_index}} \\
        \BLOCK{ for i in range(nels) }
            $L_{\VAR{i+1}} = \VAR{lengths[i]}$ \BLOCK{ if not loop.last }&\BLOCK{ endif }
        \BLOCK{ endfor }
        \\
        \BLOCK{ for i in range(nels) }
            $E_\VAR{i+1} = \VAR{materials[i]}$ \BLOCK{ if not loop.last }&\BLOCK{ endif }
        \BLOCK{ endfor }
        \\
        \BLOCK{ for i in range(nels) }
            $I_\VAR{i+1} = \VAR{inertia[i]}$ \BLOCK{ if not loop.last }&\BLOCK{ endif }
        \BLOCK{ endfor }
    \end{tabular}
    \label{tab:dane_zadania2}
    \vspace{3mm}
\end{minipage}


Na układ działają następujące obciążenia:
$P = \VAR{load.P} \si{kN}$, $M  = \VAR{load.M} \si{kNm}$, $q = \VAR{load.q} \si{kN/m}$.


\vspace{10mm}
Wykonane opracowanie powinno zawierać:

\begin{table}[ht]
    \centering
    \begin{tabular}{m{.01\textwidth}m{.7\textwidth}m{.15\textwidth}}
    \toprule
    1 & Rysunek dyskretyzacji układu & 5 pkt \\ \midrule
    2 & Skrypt rozwiązujący dane zadanie w formacie *.m & 5 pkt \\ \midrule
    3 & Wykresy sił wewnętrznych (T, M) & 5 pkt \\ \midrule
    4 & Szkic deformacji & 5 pkt \\ \bottomrule
    \end{tabular}
    \label{tab:punktacja}
\end{table}

\vspace{5mm}
UWAGI:
\begin{itemize}
    \item Rysunek dyskretyzacji oraz wykresy sił tnących, momentów i deformacji powinny
    zostać wykonane na odwrocie karty projektu.
    \item Skrypt rozwiązujący daną belkę powinen zostać wgrany do modułu projektu
    w kursie Metody Obliczenione na eNauczaniu.
\end{itemize}

\end{document}
