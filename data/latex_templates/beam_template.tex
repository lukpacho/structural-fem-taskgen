\documentclass[a4paper,10pt]{article}
\usepackage[left=25mm,right=25mm,top=25mm,bottom=25mm,bindingoffset=0mm]{geometry}
\usepackage{fontspec}
\usepackage{fancyhdr}
\usepackage{graphicx}
\usepackage[detect-all=true]{siunitx}
\usepackage{stanli}
\usepackage{amsmath}
\usepackage{booktabs}
\usepackage{tikz}
\usepackage{calc}
\usepackage{xcolor}
\usepackage{adjustbox}


\pagestyle{fancy}
\fancyhead[L]{\small Katedra Wytrzymałości Materiałów, WILiŚ, PG}
\fancyhead[R]{\small rok. akad. \VAR{academic_year}}
\fancyfoot[L]{\textcolor{gray}{\textit{Numer zadania (do informacji sprawdzającego): \VAR{beam_version}\VAR{simulation_index}}}}
\fancyfoot[C]{}

\begin{document}

    \begin{center}
    {\Large \textbf{Podstawy Mechaniki Komputerowej - Projekt}}
    \end{center}

    \begin{table}[ht]
        \centering
        \begin{tabular}{
            m{.175\textwidth}m{.215\textwidth}m{.175\textwidth}m{.175\textwidth}m{.1\textwidth}}
        \toprule
        Imię & Nazwisko & Numer albumu & Numer grupy  & Ocena \\ \midrule
             &          &              &              &       \\ \bottomrule
        \end{tabular}
        \label{tab:dane_studenta}
    \end{table}

    \noindent\textbf{Treść zadania:} \vspace{1mm}

    Dla przedstawionego układu prętowego należy:

    \begin{itemize}
        \item Wyznaczyć wykresy sił wewnętrznych: sił tnących \(T\) oraz momentów zginających \(M\), \textbf{przedstawione w odpowiednich jednostkach}.
        \item Sporządzić \textbf{rysunek} deformacji układu, oznaczając \textbf{wartości charakterystyczne ugięć} oraz uwzględniając \textbf{obliczone kąty obrotu}.
    \end{itemize}

    Zadanie należy rozwiązać, stosując \textbf{Macierzową Metodę Przemieszczeń} i element belkowy. W analizie numerycznej proszę wykorzystać bibliotekę \textbf{CalFEM} w środowisku \textbf{MATLAB}. Poniższa tabela zawiera dane niezbędne do realizacji zadania:

    \begin{table}[ht]
    \centering
    \renewcommand{\arraystretch}{1.25}
    \begin{tabular}{\VAR{"l" * (nels) + "l"}}
    \toprule
    Długość [\si{m}] &
            \BLOCK{ for i in range(nels) }
                $L_{\VAR{i+1}} = \VAR{lengths[i]}$ \BLOCK{ if not loop.last }&\BLOCK{ endif }
            \BLOCK{ endfor }
            \\
    Moduł Younga [\si{GPa}] &
            \BLOCK{ for i in range(nels) }
                $E_\VAR{i+1} = \VAR{materials[i]}$ \BLOCK{ if not loop.last }&\BLOCK{ endif }
            \BLOCK{ endfor }
            \\
    Moment bezwładności [\si{\centi\meter^4}] &
            \BLOCK{ for i in range(nels) }
                $I_\VAR{i+1} = \VAR{inertias[i]}$ \BLOCK{ if not loop.last }&\BLOCK{ endif }
            \BLOCK{ endfor }
            \\
    \bottomrule
    \end{tabular}
    \label{tab:dane_zadania2}
    \end{table}

    Na zadany układ działają następujące obciążenia w postaci
    \BLOCK{ for force in forces}
        \BLOCK{if loop.first}
            \BLOCK{ if loop.length > 1} sił skupionych
            \BLOCK{ else } siły skupionej
            \BLOCK{ endif }
        \BLOCK{ endif }
        $P_\VAR{loop.index} = \VAR{force.magnitude} \si{\kilo\newton}$%
        \BLOCK{ if not loop.last }, \BLOCK{ endif }
    \BLOCK{ endfor }
    ,
    \BLOCK{ for moment in moments}
        \BLOCK{if loop.first}
            \BLOCK{ if loop.length > 1} momentów zginających
            \BLOCK{ else } momentu zginającego
            \BLOCK{ endif }
        \BLOCK{ endif }
        $M_\VAR{loop.index} = \VAR{moment.magnitude} \si{\kilo\newton m}$
        \BLOCK{ if not loop.last }, \BLOCK{ endif }
    \BLOCK{ endfor }
    oraz
    \BLOCK{ for lineload in lineloads}
        \BLOCK{if loop.first}
            \BLOCK{ if loop.length > 1} obciążeń liniowych
            \BLOCK{ else } obciążenia liniowego
            \BLOCK{ endif }
        \BLOCK{ endif }
        $q_\VAR{loop.index} = \VAR{lineload.magnitude} \si{\kilo\newton/\meter}$%
        \BLOCK{ if not loop.last }, \BLOCK{ endif }
    \BLOCK{ endfor }
    .

    \vspace{5mm}

    \noindent\textbf{Schemat układu:}

    \begin{center}
    \begin{tikzpicture}[x=\VAR{x_scale}cm, y=\VAR{y_scale}cm]
    % nodes definition
    \BLOCK{ for node in nodes }
        \point{\VAR{node.name}}{\VAR{node.x}}{\VAR{node.y}}
    \BLOCK{ endfor }
    % elements definition
    \BLOCK{ for element in elements }
        \beam{2}{\VAR{element.start}}{\VAR{element.end}}
        \notation{4}{\VAR{element.start}}{\VAR{element.end}}[$E_\VAR{loop.index}$, $I_\VAR{loop.index}$][0.5][below=.2mm];
        \dimensioning{1}{\VAR{element.start}}{\VAR{element.end}}{-1.5}[$L_\VAR{loop.index}$]
    \BLOCK{ endfor }
    % supports definition
    \BLOCK{ for support in supports }
        \support{\VAR{support.type_number}}{\VAR{support.node}}[\VAR{support.rotation}]
    \BLOCK{ endfor }
    % hinges definition
    \BLOCK{ for hinge in hinges }
        \hinge{1}{\VAR{hinge}}
    \BLOCK{ endfor }
    % loads definition
    \BLOCK{ for force in forces}
        \load{1}{\VAR{force.node}}[\VAR{force.rotation}][\VAR{force.length}][\VAR{force.distance}]
        \node[fill=white, fill opacity=0.8, text opacity=1, inner sep=1pt] at (\VAR{force.node}) [shift={(-0.3, \VAR{force.length} + 0.3)}] {$P_\VAR{loop.index}$};
    \BLOCK{ endfor }
    \BLOCK{ for moment in moments}
        \load{\VAR{moment.orientation}}{\VAR{moment.node}}[\VAR{moment.rotation}][\VAR{moment.angle}]
        \node[fill=white, fill opacity=0.8, text opacity=1, inner sep=1pt] at (\VAR{moment.node}) [shift={(\VAR{moment.label_x_offset}, \VAR{moment.label_y_offset})}] {$M_\VAR{loop.index}$};
    \BLOCK{ endfor }
    \BLOCK{ for lineload in lineloads}
        \lineload{3}{\VAR{lineload.node1}}{\VAR{lineload.node2}}[\VAR{lineload.value}][\VAR{lineload.value}][\VAR{lineload.distance}]
        \coordinate (midpoint) at ($ (\VAR{lineload.node1})!0.5!(\VAR{lineload.node2}) $);
        \node[fill=white, fill opacity=0.8, text opacity=1, inner sep=1pt] at (midpoint) [shift={(0, \VAR{lineload.label_y_offset})}] {$q_\VAR{loop.index}$};
    \BLOCK{ endfor }
    \end{tikzpicture}
    \end{center}

    \vspace{5mm}

    \noindent\textbf{Wymagane elementy opracowania:}

    \begin{enumerate}
    \item Rysunek dyskretyzacji układu \dotfill 5 pkt
    \item Skrypt w MATLAB-ie rozwiązujący zadanie (*.m) \dotfill 5 pkt
    \item Wykresy sił wewnętrznych: siły tnące \(T\) [\si{\kilo\newton}] i momenty zginające \(M\) [\si{\kilo\newton m}] \dotfill 10 pkt
    \item Rysunek deformacji układu [\si{\centi\meter}] \dotfill 5 pkt
    \end{enumerate}

    \vspace{5mm}

    \noindent\textbf{Uwagi:}
    \begin{itemize}
    \item Rysunek dyskretyzacji, wykresy sił tnących, momentów zginających oraz rysunek deformacji należy umieścić na odwrocie karty projektowej.
    \item Skrypt rozwiązujący zadanie należy przesłać do odpowiedniego modułu na platformie eNauczanie w ramach kursu \textit{Podstawy Mechaniki Komputerowej}.
    \item Nieprzesłanie skryptu skutkuje przyznaniem 0 punktów za elementy opracowania nr 3 i 4.
    \end{itemize}

\end{document}
