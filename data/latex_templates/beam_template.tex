\documentclass[a4paper,10pt]{article}
\usepackage[left=25mm,right=25mm,top=25mm,bottom=25mm,bindingoffset=0mm]{geometry}
\usepackage{fontspec}
\usepackage{fancyhdr}
\usepackage{graphicx}
\usepackage[detect-all=true]{siunitx}
\usepackage{stanli}
\usepackage{amsmath}
\usepackage{booktabs}

\pagestyle{fancy}
\fancyhead[L]{\small Katedra Wytrzymałości Materiałów, WILiŚ, PG}
\fancyhead[R]{\small rok. akad. \VAR{academic_year}}
\fancyfoot{}

\begin{document}

    \begin{center}
    {\Large \textbf{Metody Obliczeniowe - Projekt}}
    \end{center}

    \begin{table}[ht]
        \centering
        \begin{tabular}{
            m{.175\textwidth}m{.215\textwidth}m{.175\textwidth}m{.175\textwidth}m{.1\textwidth}}
        \toprule
        Imię & Nazwisko & Numer albumu & Numer grupy  & Ocena \\ \midrule
             &          &              &              &       \\ \bottomrule
        \end{tabular}
        \label{tab:dane_studenta}
    \end{table}

    Dla danego układu prętowego należy wyznaczyć wykresy sił wewnętrznych (siły tnące \(T\) i momenty zginające \(M\))
    oraz przedstawić szkic deformacji układu. Zadanie należy rozwiązać, stosując Macierzową Metodę Przemieszczeń
    dla elementów belkowych. W analizie numerycznej proszę wykorzystać bibliotekę CalFEM w środowisku MATLAB.
    Poniższa tabela zawiera dane wymagane do realizacji zadania, przedstawione w jednostkach:
    \(L \rightarrow \si{m}\), \(E \rightarrow \si{GPa}\), \(I \rightarrow \si{\centi\meter^4}\).

    \noindent
    \begin{minipage}[s]{\textwidth}
        \vspace{3mm}
        \renewcommand{\arraystretch}{1.25}
        \centering
        \begin{tabular}{\VAR{"l" * (nels)}}
            \multicolumn{\VAR{nels}}{l}{Zestaw danych nr \VAR{beam_version}\VAR{simulation_index}} \\
            \BLOCK{ for i in range(nels) }
                $L_{\VAR{i+1}} = \VAR{lengths[i]}$ \BLOCK{ if not loop.last }&\BLOCK{ endif }
            \BLOCK{ endfor }
            \\
            \BLOCK{ for i in range(nels) }
                $E_\VAR{i+1} = \VAR{materials[i]}$ \BLOCK{ if not loop.last }&\BLOCK{ endif }
            \BLOCK{ endfor }
            \\
            \BLOCK{ for i in range(nels) }
                $I_\VAR{i+1} = \VAR{inertias[i]}$ \BLOCK{ if not loop.last }&\BLOCK{ endif }
            \BLOCK{ endfor }
        \end{tabular}
        \label{tab:dane_zadania2}
        \vspace{3mm}
    \end{minipage}


%    Na układ działają następujące obciążenia:
%    $P = \VAR{forces.P} \si{kN}$, $M  = \VAR{moments.M} \si{kNm}$, $q = \VAR{lineloads.q} \si{kN/m}$.
%    \vspace{10mm}


    \noindent\begin{minipage}[s]{\textwidth}
        \centering\begin{tikzpicture}
        % nodes definition
        \BLOCK{ for node in nodes }
            \point{\VAR{node.name}}{\VAR{node.x}}{\VAR{node.y}}
        \BLOCK{ endfor }
        % elements definition
        \BLOCK{ for element in elements }
            \beam{2}{\VAR{element.start}}{\VAR{element.end}}
            \notation{4}{\VAR{element.start}}{\VAR{element.end}}[$E_\VAR{loop.index}$, $I_\VAR{loop.index}$][0.5][below];
            \dimensioning{1}{\VAR{element.start}}{\VAR{element.end}}{-1.5}[$L_\VAR{loop.index}$]
        \BLOCK{ endfor }
        % hinges definition
        \BLOCK{ for hinge in hinges }
            \hinge{1}{\VAR{hinge}}
        \BLOCK{ endfor }
        % supports definition
        \BLOCK{ for support in supports }
            \support{\VAR{support.type_number}}{\VAR{support.node}}[\VAR{support.rotation}]
        \BLOCK{ endfor }
        % loads definition
        \BLOCK{ for force in forces}
            \load{1}{\VAR{force.node}}[\VAR{force.rotation}][\VAR{force.length}][\VAR{force.distance}]
        \BLOCK{ endfor }
        \BLOCK{ for moment in moments}
            \load{\VAR{moment.orientation}}{\VAR{moment.node}}[\VAR{moment.rotation}][\VAR{moment.angle}]
        \BLOCK{ endfor }
        \BLOCK{ for lineload in lineloads}
            \lineload{3}{\VAR{lineload.node1}}{\VAR{lineload.node2}}[\VAR{lineload.value}][\VAR{lineload.value}]
        \BLOCK{ endfor }
        \end{tikzpicture}
    \end{minipage}

    \vspace{10mm}
    Wykonane opracowanie powinno zawierać:

    \begin{table}[ht]
    \centering
    \begin{tabular}{m{.01\textwidth}m{.7\textwidth}m{.15\textwidth}}
    \toprule
    1 & Rysunek układu po dyskretyzacji & 5 pkt \\ \midrule
    2 & Skrypt w MATLAB-ie rozwiązujący zadanie (*.m) & 5 pkt \\ \midrule
    3 & Wykresy sił wewnętrznych: siły tnące \(T\) i momenty zginające \(M\) & 5 pkt \\ \midrule
    4 & Szkic deformacji układu & 5 pkt \\ \bottomrule
    \end{tabular}
    \label{tab:punktacja}
    \end{table}

    \vspace{5mm}
    UWAGI:
    \begin{itemize}
        \item Rysunek dyskretyzacji, wykresy sił tnących, momentów zginających oraz szkic deformacji należy umieścić na odwrocie karty projektowej.
        \item Skrypt rozwiązujący zadanie należy przesłać do odpowiedniego modułu na platformie eNauczanie w ramach kursu Metody Obliczeniowe.
    \end{itemize}

\end{document}
